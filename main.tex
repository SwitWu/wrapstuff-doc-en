\documentclass{l3doc}
\usepackage{wrapstuff}
\usepackage{graphicx}
\usepackage{lipsum}
\title{The \textsf{wrapstuff} package\thanks{Corresponding to version v0.3*
  of the Chinese documentation.}}
\author{Author: Qing Lee \\ Translator: Siyu Wu}
\date{\today}
\newcommand{\opt}[1]{\texttt{#1}}
\ExplSyntaxOn
\makeatletter
\DeclareDocumentCommand \ADDSPECEIALKEY { m }
  {
    \cs_gset_eq:NN \@@_codedoc_names_typeset_save:
                   \__codedoc_names_typeset:
    \cs_gset_protected:Npn \__codedoc_names_typeset:
      {
        \cs_gset_eq:NN \__codedoc_names_typeset:
                       \@@_codedoc_names_typeset_save:
        \__codedoc_names_typeset:
        #1 \\
      }
  }
\makeatother
\ExplSyntaxOff
\begin{document}

\maketitle

\section{Basic Usage}

The \pkg{wrapstuff} package provides another implementation
of wrapping text around a figure.
\pkg{wrapstuff} tries to integrate and extend the function
of some similar packages such as \pkg{picinpar}, \pkg{floatflt},
\pkg{wrapfig}, \pkg{cutwin} and \pkg{wrapfig2}.
This package is compatible with \pkg{caption}, \pkg{float}
and \pkg{floatrow}. And it tries to be compatible with display
math and various kinds of \LaTeX{} list environments so that
they can wrap figures and tables correctly.

The implementation of \pkg{wrapstuff} depends on the paragraph
hooks provided by \LaTeX{} 2021-06-01, and also depends on
\LaTeX3 2022-04-10 or later version.

\begin{function}{wrapstuff}
  \begin{syntax}
    \cs{begin}\{wrapstuff\}\oarg{options}
      \meta{wrapped contents}
    \cs{end}\{wrapstuff\}
    \meta{main text}
  \end{syntax}
\end{function}

\pkg{wrapstuff} only provides one environment called \env{wrapstuff},
which will wrap text from the paragraph after this environment on.
For example,

\begin{verbatim}
  \begin{wrapstuff}[c,top=1]
    \includegraphics[width=\dimeval{\linewidth/3}]{example-image.pdf}
  \end{wrapstuff}
  \lipsum[1-2]
\end{verbatim}

\begin{wrapstuff}[c,top=1]
  \includegraphics[width=\dimeval{\linewidth/3}]{example-image.pdf}
\end{wrapstuff}
\lipsum[1]

\begin{function}{\wrapstuffset}
  \begin{syntax}
    \cs{wrapstuff}\marg{option list}
  \end{syntax}
  The options of \pkg{wrapstuff} can be set when invoking the package,
  but they can also be set using \cs{wrapstuffset}.
\end{function}

\ADDSPECEIALKEY{\textrm{\meta{n}}}
\begin{function}{top}
  \begin{syntax}
    top = \meta{non-negative integer}
  \end{syntax}
  sets the number of the line after which \pkg{wrapstuff} begins to wrap text.
  \meta{n} is the abbreviation of setting \texttt{top} to \meta{n}.
  The initial value is 0. 
\end{function}

\begin{function}{width}
  \begin{syntax}
    width = \meta{width}
  \end{syntax}
  sets the width of wrapped contents. The initial value is 0pt,
  which represents for the natural width.
  At this time, the contents in the \env{wrapstuff} environment
  can only have one line and users are not permitted to use
  \cmd\\ to break line or \cs{par} to start new paragraph.
  If one needs to break line or start a new paragraph in the environment,
  he has to set proper value for \texttt{width}.
\end{function}

\begin{function}{height}
  \begin{syntax}
    height = \meta{height}
  \end{syntax}
  sets the height of the wrapped contents. The initial value is 0pt,
  which represents for the natural height.
\end{function}

\begin{function}{lines}
  \begin{syntax}
    lines = \meta{positive integer}
  \end{syntax}
  sets the number of lines occupied by the wrapped contents.
  The initial value is empty, which means that the number will
  be calculated automatically.
\end{function}

\begin{function}{linewidth}
  \begin{syntax}
    linewidth = \meta{width}
  \end{syntax}
  sets the line width of the main text.
  The initial value is \cs{linewidth} and typically it is unnecessary
  to adjust it.
\end{function}

\begin{function}{l,r,c,i,o,ratio}
  \begin{syntax}
    l/r/c/i/o
    ratio = \meta{real number}
  \end{syntax}
  These options set the position of the wrapped contents relative to
  the main text. Options \opt{l/r/c/i/o} means left, right, center,
  inner and outer respectively. The option \opt{ratio} sets the
  ratio of the line width occupied by the main text on the left side
  of the wrapped contents. One can set \opt{ratio} to any proper real number
  between the interval $[0,1]$.
  Options \opt{i/o} can be used together with \opt{ratio}.
  The initial value is \opt{r}.
\end{function}

\begin{function}{column}
  \begin{syntax}
    column = \meta{true|false|par}
  \end{syntax}
  controls whether to display the main text around the wrapped contents
  as two columns or not. Only when setting the option \opt{c}
  or setting \opt{ratio} in $(0,1)$ this option has its effect.
  \opt{false} means not displaying the main text in two columns
  and \opt{par} means displaying the main text as two columns
  but with per paragraph as a separate unit.
\end{function}

\begin{verbatim}
  \def\lorem{%
    Just then her head struck against the roof of the hall:
    in fact she was now more than nine feet high, and she
    at once took up the little golden key and hurried off
    to the garden door.\par
    Poor Alice! It was as much as she could do, lying down
    on one side, to look through into the garden with one eye;
    but to get through was more hopeless than ever: 
    she sat down and began to cry again.}
  
  \begin{wrapstuff}[c,column=par]
    \includegraphics[width=3cm]{example-image.pdf}
  \end{wrapstuff}
  \lorem
  
  \begin{wrapstuff}[c]
    \includegraphics[width=3cm]{example-image.pdf}
  \end{wrapstuff}
  \lorem
  
  \begin{wrapstuff}[c,column=false]
    \includegraphics[width=3cm]{example-image.pdf}
  \end{wrapstuff}
  \lorem
\end{verbatim}

\def\lorem{%
Just then her head struck against the roof of the hall: in fact she was now more than nine feet high, and she at once took up the little golden key and hurried off to the garden door.\par
Poor Alice! It was as much as she could do, lying down on one side, to look through into the garden with one eye; but to get through was more hopeless than ever: she sat down and began to cry again.}
\begin{wrapstuff}[c,column=par]
  \includegraphics[width=3cm]{example-image.pdf}
\end{wrapstuff}
\lorem

\begin{wrapstuff}[c]
  \includegraphics[width=3cm]{example-image.pdf}
\end{wrapstuff}
\lorem

\begin{wrapstuff}[c,column=false]
  \includegraphics[width=3cm]{example-image.pdf}
\end{wrapstuff}
\lorem

\begin{function}{leftsep,rightsep,hsep}
  \begin{syntax}
    leftsep = \meta{length}
    rightsep = \meta{length}
    hsep = \meta{length}
  \end{syntax}
  set the left/right space between wrapped contents and main text.
  The option \texttt{hsep} sets \texttt{leftsep} and \texttt{rightsep}
  simultaneously. The initial value is 1em.
\end{function}

\begin{function}{abovesep,belowsep,vsep}
  \begin{syntax}
    abovesep = \meta{length}
    belowsep = \meta{length}
    vsep = \meta{length}
  \end{syntax}
  set the upper/lower space between wrapped contents and main text.
  The option \texttt{vsep} sets \texttt{abovesep} and \texttt{belowsep}
  simultaneously. The initial value is 0pt.
\end{function}

\begin{function}{hoffset}
  \begin{syntax}
    hoffset = \meta{length}
  \end{syntax}
  sets the length of the wrapped contents that extend beyond
  the main text area. When \texttt{c} is used or \texttt{ratio}
  is set between $(0,1)$, this option loses effect.
  One special value \cs{width} is used to represent the total width
  of wrapped contents and the corresponding spacing.
  If setting \opt{hoffset} equals to \cs{width}, then
  the wrapped contents are totally placed outside of the main text area.
  The initial value is 0pt.
\end{function}

\begin{function}{voffset}
  \begin{syntax}
    voffset = \meta{length}
  \end{syntax}
  is used to adjust the vertical position of the wrapped contents.
  The initial value is 0pt.
\end{function}

\begin{function}{type}
  \begin{syntax}
    type = \meta{float type}
  \end{syntax}
  sets the type of float of the wrapped contents.
  The initial value is empty.
  If one needs to use \cs{caption} in \env{wrapstuff},
  then he has to set \opt{type} to \opt{figure} or \opt{table}
  and set the value for \opt{width}.
\end{function}

\begin{function}{\wrapstuffclear}
  If the number of lines in the current paragraph is not enough,
  the effect of \pkg{wrapstuff} will continue to the next paragraph.
  Sometimes it may cause some undesirable results,
  in which case one could issue \cs{wrapstuffclear} to eliminate
  the effect.
\end{function}
\end{document}